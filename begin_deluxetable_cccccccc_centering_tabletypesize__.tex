\newline
\begin{deluxetable}{cccccccc}
\centering
\tabletypesize{\footnotesize}
\tablecolumns{8} 
\tablewidth{0pt}
% \tablecaption{Cross-correlation Fit Details}
% \label{tab:lumresults1}}
 \tablehead{
 \colhead{QSO} \vspace{-0.2cm}& \colhead{$R$ Range} &  & & &  &\colhead{Separation} & \colhead{Result}\\ \vspace{-0.2cm}
  & & \colhead{$\langle f \rangle$} & \colhead{$r_0$} &  \colhead{$\gamma$}& $W$ & & \\
 \colhead{Division} & \colhead{(Mpc/h)} & \colhead{} & \colhead{} & \colhead{}&& \colhead{(\%)} & \colhead{Strength}} 
 \startdata 
 \vspace{-0.2cm} 1/3 Bright &  & $4.24 \cdot 10^{-4}$ & 6.19 && 96.97 & &\\ \vspace{-0.2cm}
  & [0.3,3] & & & 1.77 & &  96.7 & 1.9$\sigma$ \\ 
   2/3 Dim & & $4.26 \cdot 10^{-4}$ & 4.48 & & 52.77
 \enddata
 \vspace{-0.8cm}
 \tablecomments{Luminosity dependent quasar clustering using a cross-correlation technique between CS82 galaxies ($M<23.5$) and SDSS, BOSS, and 2SLAQ quasars~($0.5 < z < 1.0$). The quasars were broken up into 1/3 brightest and 2/3 dimmest... }
\end{deluxetable}