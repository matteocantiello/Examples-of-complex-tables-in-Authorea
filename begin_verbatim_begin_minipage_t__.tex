\begin{verbatim}
\begin{minipage}[t]{\textwidth}
\centering
\caption{Hover on the table headers to see what the definitions for the  different quantities.}
\label{table}
\renewcommand{\footnoterule}{}

\begin{tabular}{l c | c c c c c c  c c c c c c}
\hline\hline
   $M_{\rm{ini}}$    & $R_{\star}$    \T  & $R_{\mathrm{FeCZ}}$\footnote{Radial coordinate of the top of the FeCZ.} & $\Delta R_{\mathrm{FeCZ}}$\footnote{Radial extension of the FeCZ.}    &      $\hp$\footnote{Pressure scale height at top/bottom of the FeCZ.} & $\vca$\footnote{Maximum of the convective velocity inside the FeCZ.}  & $\rho$\footnote{Density at $\vca$.} & $\Delta M_{\rm FeCZ}$\footnote{Mass contained in the convective region.}  & $\Delta M_{\rm top}$\footnote{Mass in the radiative layer between the stellar surface and the upper boundary of the convective zone.}    & $\tau_{\rm turn}$\footnote{Convective turnover time, $\tau_{\rm turn}:= \hp/\vca$.} &   $\tau_{\rm conv}$\footnote{Time that a piece of stellar material spends 
inside a convective region, $\tau_{\rm conv}:= \Delta M_{\rm FeCZ}/\dot M$.} & $ \dot{M}$ \\
 $\mso$ &  $\rso$ \T   & $\rso$   & $\rso$ & $\rso$ &  km~s$^{-1}$     & g~cm$^{-3}$         & $\mso$             & $\mso$                            &       days          &  days     & $\mso \rm{ yr}^{-1}$ \\

\hline
20    & 10.46   &  10.20 & 0.28    & 0.08 - 0.24  &   10.74  &  $7.4\times10^{-8}$    & $3.6\times10^{-6}$     &  $5.8\times10^{-7}$       &   0.53       &  18250  &  $7.3\times10^{-8}$   \\
60    &  22.04  &   21.34 & 2.84   & 0.23 - 1.93 &   69.26  &    $6.2\times10^{-9}$  & $1.6\times10^{-5}$     &  $9.8\times10^{-7}$     &   0.61       &   1570    &  $3.7\times10^{-6}$      \\
\hline

\end{tabular}
\end{minipage}
\end{table*} 
\end{verbatim}