\begin{table*} \label{tab:tab1}

\begin{minipage}{\textwidth}
\caption{Properties of the envelope convection zones in our 20 and 60 $\mso$ models of solar metallicity. These are the same models shown in the top panel of Fig.~\ref{fig:20zsunsurf} and in Fig.~\ref{fig:60zsunsurf}. The values in the table refer
to t=\ttwenty  for the 20 $\mso$ model and t=\tsixty for the 60 $\mso$ model.}
\label{clumps}
\renewcommand{\footnoterule}{}
\centering
\begin{tabular}{lccccccccc}
\hline\hline
   M        \T  & Zone    &      $\hp$ & $\vca$  & $\Delta M_{\rm conv}$\footnote{Mass contained in the convective region.}  & $\Delta M_{\rm top}$\footnote{Mass in the radiative layer between the stellar surface and the upper boundary of the convective zone.}  & $N_{\rm cells}$\footnote{Expected number of convective cells, $N_{\rm cells}:= (R_{\star}/H_{\rm P})^2$.}  & $\tau_{\rm turn}$\footnote{Convective turnover time, $\tau_{\rm turn}:= H_{\rm P}/\vca$.} &   $\tau_{\rm conv}$\footnote{Time that a piece of stellar material spends 
inside a convective region, $\tau_{\rm conv}:= \Delta M_{\rm conv}/\dot M$.} & $ \dot{M}$ \\
 $\mso$\T  &         & $\rso$ &  km~s$^{-1}$              & $\mso$             & $\mso$           &                  &       days          &  days     & $\mso \rm{ yr}^{-1}$ \\

\hline
20              & He      & 0.025      &   0.08       & $7.6\times10^{-9}$     &  $1.9\times10^{-9}$     & $1.8\times10^{5}$ &   2.5        &  38         &      $7.3\times10^{-8} $      \\
20              & Fe      & 0.08       &   2.40       & $3.6\times10^{-6}$     &  $5.8\times10^{-7}$     & $1.8\times10^{4}$  &   0.25       &  18250  &  $7.3\times10^{-8}$   \\
60              & Fe      & 0.24       &   2.25       & $1.6\times10^{-5}$     &  $9.8\times10^{-7}$     & $8.5\times10^{3}$  &   0.83       &   1570    &  $3.7\times10^{-6}$      \\
\hline

\end{tabular}
\end{minipage}

\end{table*}